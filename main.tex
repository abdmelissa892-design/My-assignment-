
\documentclass[12pt]{article}
\usepackage[utf8]{inputenc}
\usepackage[english]{babel}
\usepackage{graphicx}
\usepackage[backend=biber, style=authoryear]{biblatex}

% This part creates your reference file automatically
\begin{filecontents}{references.bib}
@book{orwell1949,
    author = {Orwell, George},
    title = {1984},
    year = {1949},
    publisher = {Secker & Warburg}
}
@article{flores2016,
    author = {Flores, Nelson and Rosa, Jonathan},
    title = {Undoing Appropriateness: Raciolinguistic Ideologies},
    journal = {Harvard Educational Review},
    year = {2016}
}
\end{filecontents}

\addbibresource{references.bib}

\begin{document}

\begin{titlepage}
    \centering
    \vspace*{2cm}
    {\huge\bfseries Literary Analysis: Language and Power in 1984 \par}
    \vspace{1.5cm}
    {\large\bfseries Name: [YOUR NAME] \par}
    \vspace{0.5cm}
    {\large Specialization: [YOUR SPECIALIZATION] \par}
    \vfill
    \today
\end{titlepage}

\section*{Introduction}
George Orwell’s novel 1984 explores the intersection of language, power, and social control. In the extract, Syme explains "Newspeak," a language designed to limit human thought. By destroying words, the Party aims to make thoughtcrime impossible. This analysis examines how linguistic engineering serves as a tool for political domination.

\section*{Development: Linguistic Control}
The core objective of Newspeak is the diminution of the range of thought. As noted in the text, the aim is to narrow thought to prevent dissent \parencite{orwell1949}. By removing words like "freedom," the state removes the concept from the mind. This aligns with theories on how language is used as a tool of ideological power \parencite{flores2016}.

\section*{Conclusion}
Orwell demonstrates that when language is manipulated, the power to resist is lost. The death of expressive language leads to the death of identity. This serves as a warning about the importance of language maintenance against authoritarian pressure.

\printbibliography 
\end{document}



